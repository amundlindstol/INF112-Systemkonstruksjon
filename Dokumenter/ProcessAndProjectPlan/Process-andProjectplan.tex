\documentclass[norsk]{article}

\title{Process- and Projectplan}
\begin{document}
	\maketitle
\section{Process}
We decided to use Scrum as framework for our project process. 
Scrum is an agile method that is incremental and iterative. The team split the project into smaller tasks that are placed in a back-log. Then the team choose which tasks from the backlog that should be finished in a sprint. During the sprint the team members work indepentedly with their asigned taks. After the sprint the team will have a meeting where they evaluates what they did in the sprint and asign new tasks for the next sprint. 

We choose Scrum because everybody will be present and involved when we distrubute tasks for the sprint. This will help us to hand out tasks such that no one will work on the same tasks, and make sure that no tasks are left behind. Handing out spesific tasks with a deadline will also make it easier for the groupmembers to estimate how much time they have to set aside for the sprint. 

Or spints will extends over one week. We will have our Scrum meetings every monday. At these meetings we will discuss what tasks we finished in the sprint, and delegate new tasks for the next sprint. We will not have a daily scrum meeting, but day-to-day communication will happen in Slack.

We will use Gitlab to handle our backlog, by making issues, and than working on them during the sprint. 
 
 \section{Organizing}
 We will meet every monday, from 14.00 to 17.00. This meeting will mainly focus on handing out new tasks and evaluating finished tasks. But we will also work together on the task at the monday meeting, and we will be able to discuss and ask each other for help.
 Outside of the monday-meetings our comunication will be handled in slack. 
 We will use gitlab to store our common files. The repository will consist of two branches, master and develop. We will merge the develop branch into master once a week. The idea is to always have a working program in the master-branch.
  
  In an attempt to avoid conflict and proplems during the project, we will try to:
  
  \begin{itemize}
  	\item 
  	Share knowledge, such that there is always more than one person that knows how something works in case that one person is sick or quits.
  	\item 
  	Be clear on who is responsible for a task.
  	\item 
  	Set and respect deadlines.
  	\item 
  	Write readable code. We will discuss and choose a style-guide when we start writing code. 
  	\item 
  	Be open to constructive criticism.
  	\item
  	If a confict occurs, first deal with the persons involved. 
 \end{itemize}

\section{Update iteration 2}

\textbf{Process}\\
We had some problems in our execution of the Scrum model. We found it hard to understand all the small task that we had to complete to have a fully working chess game. Therefore we did not manage to define the tasks properly before we started to code. In the next iteration we will learn from our mistakes and use more time at planning ahead. We also had some problem with the flow in git-issues, and it was not always updated or taken into account. We will discuss the proper way to use git-issues to make our process more seamless in the next iteration.

\textbf{Organizing}\\
We discovered in this iteration that one meeting every week is not enough with the expected workload. So we did arrange meetings and worked together at the study hall. We have not set a specific time for an extra meeting, but we will consider this for the next iteration.

\section{Update iteration 3}
\textbf{Process}\\
Our Scrum model worked much better in this iteration. We used more time with managing the tasks and splitting them into smaller pieces. We also had stand up meetings to keep the meetings short and eficient, and that worked out great.

\textbf{Organizing}\\
We added one more meeting in the week in this iteration. So we had two meetings a week. One on monday and one on thursday. The extra meeting allowed us to work on developing togheter.

\section{Update iteration 4}
\textbf{Process}\\
In this iteration we wanted to make sure that we splitted the responsibility of the different features more evenly between the groupmembers. So we decided to split the iteration in features and make "subteams" that would be responsible for their feature. We worked more or less independent between the features, and therefore we decided to use branches for the different features, and merged it all togheter in the end.

Considering our scrum model, we did not follow the model closely as a team, but the "subteams" was in charge of splitting big task into smaller tasks and split them fair. This was very efficient because the different subteams did not have to take all of the task in every feature into cosideration.

We continued with the stand up meetings that we tried the last iteration, because these meetings were very efficient.  

\textbf{Organizing}\\
We continued with two meetings a week, as we tried last iteration.
We did have some problems with using slack as our communication tool in the earier iterations, but in this iteration the communication was much better. We also organized our gitlab reporitory differently. We used branches for the diffrent features.
\end{document}