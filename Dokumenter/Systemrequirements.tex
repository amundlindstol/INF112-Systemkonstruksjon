
\documentclass[norsk]{article}

\title{Systemrequirements}
\begin{document}
	\maketitle
Oblig 1 - inf112 - Gruppe 1
\section{Functional requirements}
\begin{itemize}
	\item 
	The Game will either be player in a web browser or as a java program.
	\item 
	It should be possible to create a new user.
	\item 
	It should be possible to log in as an existing user.
	\item 
	A player can choose to play towards a computer-player or another human.
	\item 
	The computer-player has three skill-levels.
	\item 
	The player can select what skill-level the computer should have. 
	\item 
	The computer with beginner skill-level will only consider one move forward.
	\item 
	The computer with intermediate skill-level will consider some moves forward.
	\item 
	The computer with expert skill-level will consider many moves forward.
	\item 
	The players can choose who will start with white pieces, or they can choose random.
	\item
	The board will be squared and consist of 64 tiles (8x8) and 16 white and 16 black pieces.
	\item 
	The pieces should be able to move only to their legal tiles, according to chess rules.
	\item 
	The legal moves of a piece, should be showed in a shaded color, when the piece is selected. 
	\item 
	A move is executed by pressing a piece, and then pressing a tile which is one of the legal moves for the selected piece.  
	\item 
	If a piece is selected, the player can select another piece by pressing it.
	\item
	If a piece is selected, it can be unselected.
	\item 
	The players will take turns making a move.
	\item 
	A player can only make one move in one turn. 
	\item 
	A player can see a list of the top-rated players, and their ratings.
	\item 
	A player can ask for a draw at any point of the game, but it has to be accepted by the other player.
	\item 
	A player can resign at any point of the game and will then lose the game. 
	\item
	When the game is finished, the winner will be announced.
	\item 
	When a player is finished with a game, their rank will be updated and saved.
	\item During the game, the 5 previous moves will be showed to the players.
	
\end{itemize}

\section{Non functional requirements}
\begin{itemize}
	\item 
	The system will be a fully functional and user-friendly game of chess. 
	\item 
	The system will be available in a web browser or as a java program.
	\item 
	The game will have a design that apepals to kids in primary-school.
	\item 
	Users of the system will identify themselves by a username and a password. 
	\item 
	The system will let the user play a game of chess against either another user, or a machine-player using the standard rules of chess, as described by FIDE.
	\item 
	The rules of chess should be easy to change in case of new rule regulations.
	\item 
	The machine-player will play with three levels of "cleverness". Beginner, intermediate and expert.
	\item 
	The system will contain a list of players rank according to the chess rating system used by FIDE.
	
\end{itemize}

 
\end{document}